\documentclass[portrait]{sciposter}

\usepackage{amsmath}
\usepackage{amssymb}
\usepackage{multicol}
\usepackage{sectionbox}
\usepackage[english]{babel}
%\usepackage{fancybullets}
%\usepackage{other packages you may want to use}
\usepackage{tikz}
\usetikzlibrary{snakes}
\usetikzlibrary{arrows}
\usetikzlibrary{shapes}

%\definecolor{BoxCol}{rgb}{0.9,0.9,0.9}
% uncomment for grey background to \section boxes 
% for use with default option boxedsections

%\definecolor{BoxCol}{rgb}{0.9,0.9,1}
% uncomment for light blue background to \section boxes 
% for use with default option boxedsections

%\definecolor{SectionCol}{rgb}{0,0,0.5}
% uncomment for dark blue \section text 

%sciposter: set large font for section headings
\renewcommand{\sectionsize}{\Large}

%multicols: no rule between columns
\setlength\columnseprule{0pt}


\title{Reasoning with Powerdomains in Isabelle/HOLCF}

% Note: only give author names, not institute
\author{Brian Huffman}
 
% insert correct institute name
\institute{Department of Computer Science,\\
  Portland State University, Portland, Oregon}

\email{brianh@cs.pdx.edu}  % shows author email address below institute

%\date is unused by the current \maketitle

% The following commands can be used to alter the default logo settings
%\leftlogo[0.9]{chenille}{  % defines logo to left of title (with scale factor)
\leftlogo[2.0]{psulogo}
\norightlogo
%\rightlogo[0.52]{RuGlogo}  % same but on right

% NOTE: This will require presence of files logoWenI.png and RuGlogo.png, 
% or other supported format in the current directory  
%%%%%%%%%%%%%%%%%%%%%%%%%%%%%%%%%%%%%%%%%%%%%%%%%%%%%%%%%%%%%%%%%%%%%%%%%%%%%%%%
%%% Begin of Document

\begin{document}

%define conference poster is presented at (appears as footer)
\conference{21st International Conference on Theorem Proving in Higher Order Logics,
18--21 August 2008, Montr\'eal, Qu\'ebec, Canada}

\maketitle

%%% Begin tabular environment with minipages
\begin{tabular*}{\textwidth}{@{\extracolsep{\fill}}cc}

%%% Column one
\begin{minipage}{0.310\textwidth}

\begin{abstract}
\begin{sectionbox}{}
\begin{itemize}
\item First fully-mechanized formalization of powerdomains
\item Implemented in HOLCF logic of domain theory, in the\\ Isabelle theorem prover
\item Library hides complicated implementation details
\item Proof automation for solving equalities and inequalities
\end{itemize}
\end{sectionbox}
\end{abstract}

\section{Introduction}
Powerdomains are a domain-theoretic analog of powersets, which were designed for reasoning about the semantics of nondeterministic programs.

A powerdomain provides all of the operations of a monad (as in Haskell). In addition, it provides a binary operation for making a nondeterministic choice.
\end{minipage}

&
%%% Table for columns 2, 3
\begin{tabular*}{0.655\textwidth}{@{\extracolsep{\fill}}cc}
\multicolumn{2}{c}{
\begin{minipage}{0.655\textwidth}
\section{Varieties of Powerdomains}
This is the section which will have the big combined tables, etc. blah blah blah blah blah blah blah blah blah blah blah blah blah blah blah blah blah blah blah blah blah blah blah blah blah blah blah blah blah blah blah blah blah blah blah blah blah blah blah blah blah blah blah blah blah blah

 blah blah blah blah blah blah blah blah blah blah blah blah blah blah blah blah blah blah blah blah blah blah blah blah blah blah blah blah blah blah blah blah blah blah blah blah blah blah blah blah blah blah blah blah blah blah blah blah blah blah blah blah blah blah blah blah blah blah blah

 blah blah blah blah blah blah blah blah blah blah blah blah blah blah blah blah blah blah blah blah blah blah blah blah blah blah blah blah blah blah blah blah
\end{minipage}}
\\

%%% Column two
\begin{minipage}{0.310\textwidth}
\section{Left}
text to appear under the box containing the name of the section. This is a 3-column document. You can change the number of the columns some lines above, in the {\it multicols} environment.
\end{minipage}

&
%%% Column three
\begin{minipage}{0.310\textwidth}
\section {Right}
text to appear under the box containing the name of the section. This is a 3-column document. You can change the number of the columns some lines above, in the {\it multicols} environment.
\end{minipage}

\end{tabular*}
\end{tabular*}

%%% Begin of Multicols-Enviroment
\begin{multicols}{3}

%%% Abstract
%\begin{abstract}
\begin{sectionbox}{}
\begin{itemize}
\item First fully-mechanized formalization of powerdomains
\item Implemented in HOLCF logic of domain theory, in the\\ Isabelle theorem prover
\item Library hides complicated implementation details
\item Proof automation for solving equalities and inequalities
\end{itemize}
\end{sectionbox}
%\end{abstract}

%%% -------------------------------------------------------------------------
\section{Monads for Nondeterminism}

\begin{verbatim}
class (Monad m) => MultiMonad m where
  (+|+) :: m a -> m a -> m a
\end{verbatim}

In Haskell, the list monad is often used to model nondeterministic computations.

%\begin{sectionbox}{}
\begin{verbatim}
instance MultiMonad [] where
  xs +|+ ys = xs ++ ys
\end{verbatim}
%\end{sectionbox}

\begin{sectionbox}{The first section: in a box}
\PARstart{S}{ome} text to appear under the box containing the name of the section. This is a 3-column document. You can change the number of the columns some lines above, in the {\it multicols} environment. 
\end{sectionbox}

\section{The Second Section}

\PARstart{S}{till} some text, and let's have a reference: \cite{areference}, which will appear below outside the multicolumns. Maybe you want it to appear in a column? Just move the text before the end of the multicols environment.

\section{How to compile the poster}
\PARstart{M}{ake} sure you have both {\tt a0size.sty} and {\tt sciposter.cls} in yout tex path or in the current directory, then run {\tt pdflatex} on this file. Voil\`a.

\section{Proof Automation}

\subsection{ACI Rewriting}
Foobar, can rewrite expressions to normal form by sorting elements and removing duplicates.\\
\begin{sectionbox}{}
\begin{eqnarray}
(xs\cup ys)\cup zs & = & xs\cup(ys\cup zs)\nonumber \\
ys\cup xs & = & xs\cup ys\nonumber \\
ys\cup(xs\cup zs) & = & xs\cup(ys\cup zs)\nonumber \\
xs\cup xs & = & xs\nonumber \\
xs\cup(xs\cup ys) & = & xs\cup ys\nonumber
\end{eqnarray}
\end{sectionbox}

\subsection{Solving Inequalities}
\begin{sectionbox}{}
\begin{eqnarray}
\{x\}^{\sharp}\sqsubseteq\{y\}^{\sharp} & \iff & x\sqsubseteq y\nonumber \\
xs\sqsubseteq(ys\cup^{\sharp}zs) & \iff & (xs\sqsubseteq ys)\wedge(xs\sqsubseteq zs)\nonumber \\
(xs\cup^{\sharp}ys)\sqsubseteq\{z\}^{\sharp} & \iff & (xs\sqsubseteq\{z\}^{\sharp})\vee(ys\sqsubseteq\{z\}^{\sharp})\nonumber
\end{eqnarray}
\begin{eqnarray}
\{x\}^{\flat}\sqsubseteq\{y\}^{\flat} & \iff & x\sqsubseteq y\nonumber \\
(xs\cup^{\flat}ys)\sqsubseteq zs & \iff & (xs\sqsubseteq zs)\wedge(ys\sqsubseteq zs)\nonumber \\
\{x\}^{\flat}\sqsubseteq(ys\cup^{\flat}zs) & \iff & (\{x\}^{\flat}\sqsubseteq ys)\vee(\{x\}^{\flat}\sqsubseteq zs)\nonumber
\end{eqnarray}
\begin{eqnarray}
\{x\}^{\natural}\sqsubseteq\{y\}^{\natural} & \iff & x\sqsubseteq y\nonumber \\
\{x\}^{\natural}\sqsubseteq(ys\cup^{\natural}zs) & \iff & (\{x\}^{\natural}\sqsubseteq ys)\wedge(\{x\}^{\natural}\sqsubseteq zs)\nonumber \\
(xs\cup^{\natural}ys)\sqsubseteq\{z\}^{\natural} & \iff & (xs\sqsubseteq\{z\}^{\natural})\wedge(ys\sqsubseteq\{z\}^{\natural})\nonumber
\end{eqnarray}
\end{sectionbox}

\end{multicols}

%%% Bibliography

\begin{thebibliography}{m}

\bibitem{areference}
An Author
{\em A reference}.
A paper.

\end{thebibliography}

%%% Table of figures

\begin{tabular}{ c | c | c | c }
\hline
\begin{tikzpicture}[scale=2]
  \node (U) at (0,0) {$\bot$};
  \node (A) at (-0.75,1) {$A$};
  \node (B) at (0.75,1) {$B$};
  \draw (A) -- (U) -- (B);
\end{tikzpicture}  
&
% upper powerdomain of lifted 2-element type
\begin{tikzpicture}[scale=4]
  \node (U) at (3,0)
    {$\begin{array}{c}\{\bot,A,B\}^\sharp\\
      \{\bot,A\}^\sharp\ \{\bot,B\}^\sharp\\
      \{\bot\}^\sharp\end{array}$};
  \node (AB) at (3,1) {$\{A,B\}^\sharp$};
  \node (A) at (2,2) {$\{A\}^\sharp$};
  \node (B) at (4,2) {$\{B\}^\sharp$};
  \draw (U) -- (AB);
  \draw (A) -- (AB) -- (B);
\end{tikzpicture}  
&
% lower powerdomain of lifted 2-element type
\begin{tikzpicture}[scale=4]
  \node (U) at (0,0) {$\{\bot\}^\flat$};
  \node (UA) at (-1,1)
    {$\begin{array}{c}\{A\}^\flat \\ \{\bot,A\}^\flat\end{array}$};
  \node (UAB) at (0,2)
    {$\begin{array}{c}\{A,B\}^\flat \\ \{\bot,A,B\}^\flat \end{array}$};
  \node (UB) at (1,1)
    {$\begin{array}{c}\{B\}^\flat \\ \{\bot,B\}^\flat\end{array}$};
  \draw (U) -- (UA) -- (UAB);
  \draw (U) -- (UB) -- (UAB);
\end{tikzpicture}  
&
% convex powerdomain of lifted 2-element type
\begin{tikzpicture}[scale=4]
  \node (U) at (0,0) {$\{\bot\}^\natural$};
  \node (UA) at (-1,1) {$\{\bot,A\}^\natural$};
  \node (UB) at (1,1) {$\{\bot,B\}^\natural$};
  \node (UAB) at (0,2) {$\{\bot,A,B\}^\natural$};
  \node (A) at (-1.3,2) {$\{A\}^\natural$};
  \node (B) at (1.3,2) {$\{B\}^\natural$};
  \node (AB) at (0,3) {$\{A,B\}^\natural$};
  \draw (U) -- (UA) -- (A);
  \draw (U) -- (UB) -- (B);
  \draw (UA) -- (UAB);
  \draw (UB) -- (UAB);
  \draw (UAB) -- (AB);
\end{tikzpicture}
\\
\hline
% 3-element lattice
\begin{tikzpicture}[scale=2]
  \node (A) at (0,0) {$0$};
  \node (B) at (0,1) {$1$};
  \node (C) at (0,2) {$2$};
  \draw (A) -- (B) -- (C);
\end{tikzpicture}  
&
% upper powerdomain of 3-element lattice
\begin{tikzpicture}[scale=4]
  \node (A) at (0,0)
    {$\begin{array}{c}
      \{0,1,2\}^\sharp \\
      \{0,1\}^\sharp\ \{0,2\}^\sharp \\
      \{0\}^\sharp
      \end{array}$};
  \node (B) at (0,2)
    {$\begin{array}{c} \{1,2\}^\sharp \\ \{1\}^\sharp \end{array}$};
  \node (C) at (0,4) {$\{2\}^\sharp$};
  \draw (A) -- (B) -- (C);
\end{tikzpicture}  
&
% lower powerdomain of 3-element lattice
\begin{tikzpicture}[scale=4]
  \node (A) at (0,0) {$\{0\}^\flat$};
  \node (B) at (0,2)
    {$\begin{array}{c} \{1\}^\flat \\ \{0,1\}^\flat \end{array}$};
  \node (C) at (0,4)
    {$\begin{array}{c}
      \{2\}^\flat \\
      \{1,2\}^\flat\ \{0,2\}^\flat \\
      \{0,1,2\}^\flat
      \end{array}$};
  \draw (A) -- (B) -- (C);
\end{tikzpicture}  
&
% convex powerdomain of 3-element lattice
\begin{tikzpicture}[scale=4]
  \node (A) at (0,0) {$\{0\}^\natural$};
  \node (B) at (0,2) {$\{1\}^\natural$};
  \node (C) at (0,4) {$\{2\}^\natural$};
  \node (AB) at (1,1) {$\{0,1\}^\natural$};
  \node (BC) at (1,3) {$\{1,2\}^\natural$};
  \node (AC) at (2,2)
    {$\begin{array}{c} \{0,2\}^\natural \\ \{0,1,2\}^\natural \end{array}$};
  \draw (A) -- (B) -- (C);
  \draw (A) -- (AB) -- (B) -- (BC) -- (C);
  \draw (AB) -- (AC) -- (BC);
\end{tikzpicture}
\\
\end{tabular}

\end{document}

