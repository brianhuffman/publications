\documentclass[portrait]{sciposter}

\usepackage{amsmath}
\usepackage{amssymb}
\usepackage{multicol}
\usepackage{sectionbox}
\usepackage[english]{babel}
%\usepackage{fancybullets}
%\usepackage{other packages you may want to use}
\usepackage{tikz}
\usetikzlibrary{snakes}
\usetikzlibrary{arrows}
\usetikzlibrary{shapes}

%\definecolor{BoxCol}{rgb}{0.9,0.9,0.9}
% uncomment for grey background to \section boxes 
% for use with default option boxedsections

%\definecolor{BoxCol}{rgb}{0.9,0.9,1}
% uncomment for light blue background to \section boxes 
% for use with default option boxedsections

%\definecolor{SectionCol}{rgb}{0,0,0.5}
% uncomment for dark blue \section text 

%sciposter: set large font for section headings
\renewcommand{\sectionsize}{\Large}

%multicols: no rule between columns
\setlength\columnseprule{0pt}


\title{Reasoning with Powerdomains in Isabelle/HOLCF}

% Note: only give author names, not institute
\author{Brian Huffman}
 
% insert correct institute name
\institute{Department of Computer Science,\\
  Portland State University, Portland, Oregon}

\email{brianh@cs.pdx.edu}  % shows author email address below institute

%\date is unused by the current \maketitle

% The following commands can be used to alter the default logo settings
%\leftlogo[0.9]{chenille}{  % defines logo to left of title (with scale factor)
\leftlogo[2.0]{psulogo}
\norightlogo
%\rightlogo[0.52]{RuGlogo}  % same but on right

% NOTE: This will require presence of files logoWenI.png and RuGlogo.png, 
% or other supported format in the current directory  
%%%%%%%%%%%%%%%%%%%%%%%%%%%%%%%%%%%%%%%%%%%%%%%%%%%%%%%%%%%%%%%%%%%%%%%%%%%%%%%%
%%% Begin of Document

\begin{document}

%define conference poster is presented at (appears as footer)
\conference{21st International Conference on Theorem Proving in Higher Order Logics, Montr\'eal, Qu\'ebec, Canada, 18--21 August 2008}

\maketitle

%%% Begin of Multicols-Enviroment
\begin{multicols}{3}

%%% Abstract
\begin{abstract}
\begin{sectionbox}{}
\begin{itemize}
\item First fully-mechanized formalization of powerdomains
\item Implemented in HOLCF logic of domain theory, in the\\ Isabelle theorem prover
\item Library hides complicated implementation details
\item Proof automation for solving equalities and inequalities
\end{itemize}
\end{sectionbox}
\end{abstract}

%%% -------------------------------------------------------------------------
\section{Introduction}
\begin{itemize}
\item Powerdomains are a domain-theoretic analog of powersets, which were designed for reasoning about the semantics of nondeterministic programs.
\item A powerdomain provides all of the operations of a monad. In addition, it provides a binary operation for making a nondeterministic choice.
\item As part of domain theory, we can freely combine nondeterminism with higher-order functions and arbitrary recursion.
\end{itemize}

%%% -------------------------------------------------------------------------
\section{Lists for Nondeterminism}

In Haskell, the \textbf{lazy list monad} is often used to model nondeterministic computations. The \textbf{append} function models nondeterministic choice.

\begin{verbatim}
incdec :: Int -> [Int]
incdec x = return (x+1) ++ return (x-1)
-- increment OR decrement the argument x
\end{verbatim}

Lists are useful because they are \textbf{executable}. But they are not a good \textbf{denotational model for nondeterminism}, for several reasons:

\begin{itemize}
\item \textbf{Append is not commutative.} The following program produces the same set of results as \texttt{incdec}; ordering should not matter.
\begin{verbatim}
decinc :: Int -> [Int]
decinc x = return (x-1) ++ return (x+1)
\end{verbatim}

\item \textbf{Append is not idempotent.} The following two programs have the same set of possible results; repetition should not matter.
\begin{verbatim}
prog1, prog2 :: [Int]
prog1 = incdec 5 ++ incdec 3
prog2 = return 6 ++ return 4 ++ return 2
\end{verbatim}

\item \textbf{Append favors its left argument.} In case the left argument is partial or infinite, the right argument is ignored. In the recursive function \texttt{f} below, the \texttt{(x-1)} branch is never reached. In fact, \texttt{f} and \texttt{g} are equivalent!
\begin{verbatim}
f, g :: Int -> [Int]
f x = return x ++ f (x+1) ++ f (x-1)
g x = return x ++ g (x+1)
\end{verbatim}
\end{itemize}

The definition of powerdomain in the next section addresses all of these limitations of the list monad.

%%% -------------------------------------------------------------------------
\section{Axiomatization of Powerdomains}
A \textbf{powerdomain} is a monad with a \textbf{nondeterministic choice} operator, which is \textbf{associative}, \textbf{commutative}, and \textbf{idempotent}.

\subsection*{Powerdomain Operations}
\begin{sectionbox}{}
\begin{itemize}
\item Singleton (i.e.~monadic unit/return) $\{-\}$
\item Binary choice operator $-\uplus-$
\item Monadic bind operator $- \gg\!\!= -$
\end{itemize}
\end{sectionbox}

All operations must be \textbf{continuous}.

%%% Start Column 2
\columnbreak

\subsection*{Powerdomain Laws}
\begin{sectionbox}{}
\begin{enumerate}
\item Left unit: $\{x\} \gg\!\!= f = f(x)$
\item Right unit: $xs \gg\!\!= (\lambda x.\ \{x\}) = xs$
\item Bind-assoc: $(xs \gg\!\!= f) \gg\!\!= g = xs \gg\!\!= (\lambda x.\ f(x) \gg\!\!= g)$
\item Bind-plus: $(xs \uplus ys) \gg\!\!= f = (xs \gg\!\!= f) \uplus (ys \gg\!\!= f)$
\item Plus-assoc: $(xs \uplus ys) \uplus zs = xs \uplus (ys \uplus zs)$
\item Plus-comm: $xs \uplus ys = ys \uplus xs$
\item Plus-idem: $xs \uplus xs = xs$
\end{enumerate}
\end{sectionbox}

Laws 1--3 are the standard \textbf{monad laws} from Haskell.

The lazy list monad with append satisfies only Laws 1--5.

%%% -------------------------------------------------------------------------
%\section{Varieties of Powerdomains}

\section{Convex Powerdomain}
Given an element domain $D$, we can define the convex powerdomain $P^\natural(D)$ as a \textbf{free domain-algebra}:

\begin{sectionbox}{}
\begin{enumerate}
\item Define a \textbf{recursive datatype}, with $\{-\}^\natural$ and $-\uplus^\natural-$ as \textbf{constructors}.
\item \textbf{Quotient} this datatype modulo \textbf{associativity}, \textbf{commutativity}, and \textbf{idempotence} of $-\uplus^\natural-$ (Laws 5--7).
\item Use Laws 1 and 4 as \textbf{defining equations} for bind.
\item Prove Laws 2 and 3 by \textbf{induction} over $xs$.
\end{enumerate}
Thus $P^\natural(D)$ satisfies all seven powerdomain laws.
\end{sectionbox}

\begin{itemize}
\item $P^\natural(D)$ is \textbf{universal} in a category-theoretical sense
\item Unique powerdomain \textbf{homomorphism} from $P^\natural(D)$ to any other powerdomain of $D$
\item $P^\natural(D)$ \textbf{distinguishes} as many values as possible
\item $P^\natural(D)$ \textbf{identifies} computations whose sets of results have the same \textbf{convex-closure} (see Figure 1)
\end{itemize}

\section{Upper Powerdomain}

We can define the upper powerdomain $P^\sharp(D)$ as a free domain-algebra satisfying an \textbf{additional law}:
$$xs \uplus^\sharp ys \sqsubseteq xs$$

\begin{itemize}
\item $xs \uplus^\sharp ys$ is \textbf{greatest lower bound} of $xs$ and $ys$
\item $xs \sqsubseteq ys$ if $ys$ has \textbf{fewer} possible outcomes than $xs$
\item Binary choice is \textbf{strict}: $\bot\uplus^\sharp xs = \bot$
\item \textbf{Demonic nondeterminism}: ``Possibly not terminating is just as bad as never terminating''
\item Good for reasoning about \textbf{total correctness}
\item $P^\sharp(D)$ \textbf{identifies} computations whose sets of results have the same \textbf{upward-closure}
\end{itemize}

\section{Lower Powerdomain}

We can define the lower powerdomain $P^\flat(D)$ as a free domain-algebra satisfying an \textbf{additional law}:
$$xs \sqsubseteq xs \uplus^\flat ys$$

\begin{itemize}
\item $xs \uplus^\flat ys$ is \textbf{least upper bound} of $xs$ and $ys$
\item $xs \sqsubseteq ys$ if $ys$ has \textbf{more} possible outcomes than $xs$
\item $\bot$ is \textbf{identity} for binary choice: $\bot\uplus^\flat xs = xs$
\item \textbf{Angelic nondeterminism}: ``I don't care about execution paths that don't terminate''
\item Good for reasoning about \textbf{partial correctness}
\item $P^\flat(D)$ \textbf{identifies} computations whose sets of results have the same \textbf{downward-closure}
\end{itemize}

%%% Start Column 3
\columnbreak

%%% -------------------------------------------------------------------------
\section{Visualizing Powerdomains}

\begin{figure}
\begin{tabular*}{\columnwidth}{@{\extracolsep{\fill}}|c|c|c|}
\hline
% upper powerdomain of 3-element lattice
\begin{tikzpicture}[thick,scale=2.8]
  \node (A) at (0,0)
    {$\begin{array}{c}
      \{0,1,2\}^\sharp \\
      \{0,1\}^\sharp\ \{0,2\}^\sharp \\
      \{0\}^\sharp
      \end{array}$};
  \node (B) at (0,2)
    {$\begin{array}{c} \{1,2\}^\sharp \\ \{1\}^\sharp \end{array}$};
  \node (C) at (0,4) {$\{2\}^\sharp$};
  \draw (A) -- (B) -- (C);
\end{tikzpicture}  
&
% lower powerdomain of 3-element lattice
\begin{tikzpicture}[thick,scale=2.8]
  \node (A) at (0,0) {$\{0\}^\flat$};
  \node (B) at (0,2)
    {$\begin{array}{c} \{1\}^\flat \\ \{0,1\}^\flat \end{array}$};
  \node (C) at (0,4)
    {$\begin{array}{c}
      \{2\}^\flat \\
      \{1,2\}^\flat\ \{0,2\}^\flat \\
      \{0,1,2\}^\flat
      \end{array}$};
  \draw (A) -- (B) -- (C);
\end{tikzpicture}  
&
% convex powerdomain of 3-element lattice
\begin{tikzpicture}[thick,scale=2.8]
  \node (A) at (0,0) {$\{0\}^\natural$};
  \node (B) at (0,2) {$\{1\}^\natural$};
  \node (C) at (0,4) {$\{2\}^\natural$};
  \node (AB) at (1,1) {$\{0,1\}^\natural$};
  \node (BC) at (1,3) {$\{1,2\}^\natural$};
  \node (AC) at (2,2)
    {$\begin{array}{c} \{0,2\}^\natural \\ \{0,1,2\}^\natural \end{array}$};
  \draw (A) -- (B) -- (C);
  \draw (A) -- (AB) -- (B) -- (BC) -- (C);
  \draw (AB) -- (AC) -- (BC);
\end{tikzpicture}
\\
\hline
\end{tabular*}
\caption{Upper, lower, and convex powerdomains of the three-element linearly ordered domain $0 \sqsubseteq 1 \sqsubseteq 2$}
\end{figure}

\begin{figure}
\begin{tabular*}{\columnwidth}{@{\extracolsep{\fill}}|c|c|}
\hline
% upper powerdomain of lifted 2-element type
\begin{tikzpicture}[thick,scale=3.9]
  \node (U) at (0,0)
    {$\begin{array}{c}\{\bot,A,B\}^\sharp\\
      \{\bot,A\}^\sharp\ \{\bot,B\}^\sharp\\
      \{\bot\}^\sharp\end{array}$};
  \node (AB) at (0,1) {$\{A,B\}^\sharp$};
  \node (A) at (-0.8,1.8) {$\{A\}^\sharp$};
  \node (B) at (0.8,1.8) {$\{B\}^\sharp$};
  \draw (U) -- (AB);
  \draw (A) -- (AB) -- (B);
\end{tikzpicture}  
&
% lower powerdomain of lifted 2-element type
\begin{tikzpicture}[thick,scale=3.9]
  \node (U) at (0,0) {$\{\bot\}^\flat$};
  \node (UA) at (-1,1)
    {$\begin{array}{c}\{A\}^\flat \\ \{\bot,A\}^\flat\end{array}$};
  \node (UAB) at (0,2)
    {$\begin{array}{c}\{A,B\}^\flat \\ \{\bot,A,B\}^\flat \end{array}$};
  \node (UB) at (1,1)
    {$\begin{array}{c}\{B\}^\flat \\ \{\bot,B\}^\flat\end{array}$};
  \draw (U) -- (UA) -- (UAB);
  \draw (U) -- (UB) -- (UAB);
\end{tikzpicture}
\\ \hline
% convex powerdomain of lifted 2-element type
\multicolumn{2}{|c|}{
\begin{tikzpicture}[thick,xscale=4,yscale=3.5]
  \node (U) at (0,0) {$\{\bot\}^\natural$};
  \node (UA) at (-1,1) {$\{\bot,A\}^\natural$};
  \node (UB) at (1,1) {$\{\bot,B\}^\natural$};
  \node (UAB) at (0,2) {$\{\bot,A,B\}^\natural$};
  \node (A) at (-1.3,1.7) {$\{A\}^\natural$};
  \node (B) at (1.3,1.7) {$\{B\}^\natural$};
  \node (AB) at (0,2.7) {$\{A,B\}^\natural$};
  \draw (U) -- (UA) -- (A);
  \draw (U) -- (UB) -- (B);
  \draw (UA) -- (UAB);
  \draw (UB) -- (UAB);
  \draw (UAB) -- (AB);
\end{tikzpicture}}
\\ \hline
\end{tabular*}
\caption{Upper, lower, and convex powerdomains of the lifted two-element type
% Lifted 2-element type
\begin{tikzpicture}[thick,xscale=1.7,baseline=0.5]
  \node (U) at (0,0) {$\bot$};
  \node (A) at (-1,1) {$A$};
  \node (B) at (1,1) {$B$};
  \draw (A) -- (U) -- (B);
\end{tikzpicture}}
\end{figure}

%%% -------------------------------------------------------------------------
\section{Proof Automation}

\subsection*{ACI Rewriting}
Isabelle can use \textbf{permutative rewrite rules} to sort elements and remove duplicates in powerdomain expressions.
\begin{displaymath}
\begin{array}{rcl}
(xs\uplus ys)\uplus zs & = & xs\uplus(ys\uplus zs) \\
ys\uplus xs & = & xs\uplus ys \\
ys\uplus(xs\uplus zs) & = & xs\uplus(ys\uplus zs) \\
xs\uplus xs & = & xs \\
xs\uplus(xs\uplus ys) & = & xs\uplus ys \\
\end{array}
\end{displaymath}

\begin{itemize}
\item ACI rules can solve goals like $\{z,y,x,x,y\}^\sharp = \{x,y,z\}^\sharp$
\end{itemize}

\subsection*{Solving Inequalities}
Rewrite rules can reduce inequalities on \textbf{powerdomains} to inequalities on the \textbf{element type}.
\begin{displaymath}
\begin{array}{rcl}
\{x\}^{\sharp}\sqsubseteq\{y\}^{\sharp} & \iff & x\sqsubseteq y \\
xs\sqsubseteq(ys\uplus^{\sharp}zs) & \iff & (xs\sqsubseteq ys)\wedge(xs\sqsubseteq zs) \\
(xs\uplus^{\sharp}ys)\sqsubseteq\{z\}^{\sharp} & \iff & (xs\sqsubseteq\{z\}^{\sharp})\vee(ys\sqsubseteq\{z\}^{\sharp})\\
\\
\{x\}^{\flat}\sqsubseteq\{y\}^{\flat} & \iff & x\sqsubseteq y \\
(xs\uplus^{\flat}ys)\sqsubseteq zs & \iff & (xs\sqsubseteq zs)\wedge(ys\sqsubseteq zs) \\
\{x\}^{\flat}\sqsubseteq(ys\uplus^{\flat}zs) & \iff & (\{x\}^{\flat}\sqsubseteq ys)\vee(\{x\}^{\flat}\sqsubseteq zs)\\
\\
\{x\}^{\natural}\sqsubseteq\{y\}^{\natural} & \iff & x\sqsubseteq y \\
\{x\}^{\natural}\sqsubseteq(ys\uplus^{\natural}zs) & \iff & (\{x\}^{\natural}\sqsubseteq ys)\wedge(\{x\}^{\natural}\sqsubseteq zs) \\
(xs\uplus^{\natural}ys)\sqsubseteq\{z\}^{\natural} & \iff & (xs\sqsubseteq\{z\}^{\natural})\wedge(ys\sqsubseteq\{z\}^{\natural})
\end{array}
\end{displaymath}

\begin{itemize}
\item The inequality rewrite rules can reduce subgoals like $\{x,y\}^\sharp\sqsubseteq\{y,z\}^\sharp$ to $x \sqsubseteq z \vee y \sqsubseteq z$
\item With \textbf{antisymmetry}, these rules can solve equalities too
\end{itemize}
%%% Bibliography
%\begin{thebibliography}{m}
%\bibitem{areference}
%An Author
%{\em A reference}.
%A paper.
%\end{thebibliography}

\end{multicols}

%%% Table of figures

% 3-element lattice
%\begin{tikzpicture}[scale=2]
%  \node (A) at (0,0) {$0$};
%  \node (B) at (0,1) {$1$};
%  \node (C) at (0,2) {$2$};
%  \draw (A) -- (B) -- (C);
%\end{tikzpicture}  

\end{document}

